\documentclass[a4paper, 11pt]{article}

\usepackage[%
backend = biber,
sortlocale=de_DE,
natbib=true,
style = numeric,
autocite = superscript
]{biblatex}
\addbibresource{references.bib}
\DeclareCiteCommand{\supercite}[\mkbibsuperscript]{}
{\bibopenbracket%
	\usebibmacro{citeindex}%
	\usebibmacro{cite}%
	\usebibmacro{postnote}%
	\bibclosebracket}
{\supercitedelim}
{}

\usepackage{fancyhdr}
\pagestyle{fancy}
\fancyhf{}
\renewcommand{\sectionmark}[1]{\markboth{#1}{}}
\fancyhead[ER]{\rightmark}
\fancyhead[OL]{\nouppercase{\leftmark}}
\cfoot{\thepage}

\usepackage{float}
\newfloat{lstfloat}{htb}{lop}

\usepackage[ngerman]{isodate, babel}

\usepackage{hyperref}
\addto\extrasngerman{
	\renewcommand{\sectionautorefname}{Kapitel}
	\renewcommand{\subsectionautorefname}{Abschnitt}
	\renewcommand{\subsubsectionautorefname}{Unterabschnitt}
}

\usepackage[T1]{fontenc}
\usepackage[utf8]{inputenc}

\overfullrule=1mm

%opening
\title{Dokumentation zur Projektarbeit digitale Kaffeekasse}
\author{Jan Müller}
\date{\today}

\begin{document}

\maketitle

\section{Einleitung}
\label{sec:intro}

Die Entwicklung von Android Apps zeichnet sich unter anderem dadurch aus, dass die \textit{API Level}\footnote{Zum Zeitpunkt dieser Ausarbeitung existieren 29 API Level.} des \textit{Android Frameworks} zum Teil unterschiedlich behandelt werden müssen.
Die Versionsverteilung bei Android erstreckt sich über mehrere Versionen\footnote{https://www.bidouille.org/misc/androidcharts}.
Zur Maximierung der potenziellen Nutzer ist es entsprechend notwendig auch ältere API Level zu unterstützen.
Damit Entwickler dort ebenfalls neue Funktionen verwenden können, wurden die \textit{Support Libraries}\footnote{https://developer.android.com/topic/libraries/support-library} eingeführt.
Aus diesen entwickelten sich die \textit{Android Jetpack}\footnote{https://developer.android.com/jetpack/androidx} Bibliotheken.

Ziel dieses Projektes war die Entwicklung einer Kaffeekassen App und einer dazugehörigen Serveranwendung mit der Programmiersprache Kotlin.
Bei der Entwicklung der App sollten Android Jetpack Bibliotheken verwendet und mit bisherigen Methoden verglichen werden.
Zuerst wurde jedoch die Serveranwendung, das sogenannte \textit{Backend}, entwickelt.
%------------------------------------------------------------------------------------------------------------------------------------------------------

\section{Android App}
\label{sec:app}

\subsection{Architektur}
\label{subsec:app:architecture}

%TODO Bild
Die App verwendet die von Google empfohlene Architektur\footnote{https://developer.android.com/jetpack/guide}.
Fragmente, also Teile der Benutzeroberfläche, verwenden \textit{ViewModels} um Daten zu beziehen und auf \textit{Business Logic} zuzugreifen.
ViewModels greifen auf wiederum auf \textit{Repositories} zu, die als \textit{Single source of truth} dienen.
Das heißt der Zustand von Datenmodellen wird ausschließlich von Repositories festgelegt, wodurch inkonsistente Datenbestände vermieden werden.
%------------------------------------------------------------------------------------------------------------------------------------------------------

\subsection{Android Jetpack}
\label{subsec:app:jetpack}

\subsubsection{Core und AppCompat}
\label{subsubsec:app:jetpack:base}

Die \textit{Core} und \textit{AppCompat} Bibliotheken ermöglichen eine Verwendung von Funktionen neuer Android API Levels auf älteren Android Versionen.
Dazu stellen sie eine Vielzahl an Hilfsklassen zur Verfügung, welche die unterschiedliche Behandlung der API Levels verkapseln und die Entwicklung so vereinheitlichen und vereinfachen.

\subsubsection{Fragment}
\label{subsubsec:app:jetpack:fragment}

\subsubsection{ViewModel}
\label{subsubsec:app:jetpack:viewmodel}

\subsubsection{LiveData}
\label{subsubsec:app:jetpack:livedata}

\subsubsection{Navigation}
\label{subsubsec:app:jetpack:navigation}

\subsubsection{Room}
\label{subsubsec:app:jetpack:room}

\subsubsection{Work}
\label{subsubsec:app:jetpack:work}

\subsubsection{Erweiterungen}
\label{subsubsec:app:jetpack:extensions}

\section{Serveranwendung}
\label{sec:backend}

\subsection{Architektur}
\label{subsec:backend:architecture}

\subsection{Endpunkte}
\label{subsec:backend:endpoints}

\subsection{Authentifizierung}
\label{subsec:backend:authentication}

\section{Fazit}
\label{sec:conclusion}

%TODO italics
Die Verwendung von Android Jetpack Bibliotheken im Zusammenspiel mit Kotlin vereinfacht die Entwicklung robuster und performanter Android Apps deutlich.
Besonders die automatisierten Lebenszyklen von ViewModel und LiveData helfen bei der Vermeidung Leaks.
%------------------------------------------------------------------------------------------------------------------------------------------------------

\end{document}
