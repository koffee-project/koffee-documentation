\section{Einleitung}
\label{sec:intro}
Die Entwicklung von Android Apps zeichnet sich unter anderem dadurch aus, dass die \textit{API Level}\footnote{Zum Zeitpunkt dieser Ausarbeitung existieren 29 API Level.} des \textit{Android Frameworks} unterschiedliche Funktionalitäten bereitstellen.
Die Nutzerverteilung erstreckt sich über mehrere Versionen \autocite{androidhistory}, weshalb es zur Maximierung der potenziellen Nutzer notwendig ist auch ältere API Level zu unterstützen.
Damit Entwickler trotzdem neue Funktionen einheitlich verwenden können, wurden die \textit{Support Libraries} eingeführt \autocite{supportlibraries}.
Aus diesen wurde \textit{Android Jetpack} entwickelt \autocite{androidjetpack}, welches Thema dieser Projektarbeit ist.

Ziel dieses Projektes war die Entwicklung einer Kaffeekassen App und einer dazugehörigen Serveranwendung mit der Programmiersprache Kotlin.
Für die Entwicklung der App sollten Android Jetpack Bibliotheken verwendet werden, um ihre Vorzüge und Eigenschaften kennenzulernen.

Diese Dokumentation stellt Funktionalität, Architektur, verwendete Bibliotheken und Eigenschaften der digitalen Kaffeekassen App sowie der zugehörigen Serveranwendung vor.
Anschließend werden Entstehungsprozess beider Programme erläutert, Ergebnisse vorgestellt und ein Fazit zu Android Jetpack gezogen.
