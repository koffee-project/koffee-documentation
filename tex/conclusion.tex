\section{Fazit}
\label{sec:conclusion}
Die Vorzüge der Android Jetpack Bibliotheken zeigten sich bereits während der Entwicklung.
Mit Fragmenten und ViewModels war es möglich neue Bildschirme schnell zu implementieren.
Zudem ermöglichten Android Jetpacks visuelle Navigationsgraphen ein einfaches und übersichtliches Verknüpfen von Fragmenten.
Der dadurch gewonnene Gesamtüberblick resultierte in einer sicheren und fehlerresistenten Struktur.
Der automatisierte Lebenszyklus vieler Jetpack Komponenten erlaubte zudem eine Minimierung an Interaktionen mit dem Android Framework.
Stattdessen war es möglich den Fokus auf das Verhalten von Benutzeroberflächen und Geschäftslogik zu legen.
Des Weiteren hat sich Room als eine der wichtigsten Komponenten von Android Jetpack erwiesen.
Bei der Entwicklung trivialisierte sie Datenbankzugriffe und reduzierte Datenverwaltung auf ein Minimum.
Sichtbar wurden diese Vorzüge besonders bei der Entwicklungszeit.
Meilensteine wurden meist Wochen zuvor erreicht, wodurch zusätzliche Funktionen implementiert werden konnten.

Auch bei Verwendung der App sind Vorzüge der Jetpack Bibliotheken erkennbar.
Die Asynchronität der Kombination von Room, LiveData und ViewModels stellt sicher, dass Datenbankzugriffe nicht zu einer Blockierung der Benutzeroberfläche führen.
RecyclerViews sorgen zudem für eine performante Darstellung von Listen und ermöglichen flüssiges Scrollen durch diese.
Die Paging Bibliothek begünstigt dies weiter.
Sie sorgt dafür, dass das Laden von Listen aus der Datenbank auch bei vielen Einträgen nicht langsamer wird.
Lifecycle Komponenten sorgen zudem dafür, dass sie nicht länger als benötigt aktiv bleiben, wodurch Speicher- sowie Prozessorverbrauch reduziert werden.

Mit Android Jetpack Bibliotheken entwickelte Apps sind zwar nicht zwingend schneller oder besser als Herkömmliche, jedoch ist es für Entwickler einfacher diese Ziele zu erreichen, da Grundbausteine und Architektur bereits gegeben sind.
Weil sie zugleich eine beschleunigte Entwicklung ermöglichen und die Zukunft der von Google entwickelten Android Bibliotheken bilden, ist es empfehlenswert bei der App Entwicklung auf Jetpack Komponenten zurückzugreifen.
%------------------------------------------------------------------------------------------------------------------------------------------------------
