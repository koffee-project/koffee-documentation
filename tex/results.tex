\section{Ergebnisse}
\label{sec:results}
Alle Grundfunktionen des Kaffeekassensystem wurden sowohl in der App als auch im Backend umgesetzt.
Dazu gehören die administrative Verwaltung von Artikeln und Nutzern sowie Kaufen und Stornieren.

Nach Wünschen der Projektbetreuer wurden Benutzeroberflächen umgebaut und die Verwendung der App angepasst.
Zusätzlich wurde ein alternativer Modus für eine geteilte Verwendung durch mehrere Nutzer in der App implementiert.
Aus zeitlichen Gründen war es nicht möglich Benachrichtigungen zu implementieren, welche ausgewählte Nutzer erhalten sollten, sobald Artikelbestände unter Schwellwerte fallen.
Dabei handelte es sich um eine optionale Funktionalität.

Für das Backend wurde zusätzlich eine Konfiguration erstellt, die einen sicheren Betrieb über HTTPS ermöglicht und diesen selbstständig einrichtet.
Des Weiteren resultiert die Modulare Struktur des Backends in einer einfachen Erweiterbarkeit, falls ein Hinzufügen neuer Funktionalitäten erwünscht ist.
Die vollständige Dokumentation des Quelltexts und das Einhalten von Formatierungskonventionen unterstützen diese Eigenschaft weiter.
%------------------------------------------------------------------------------------------------------------------------------------------------------
