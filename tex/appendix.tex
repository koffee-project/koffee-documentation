\begin{appendix}
\appendixpage

\chapter{Inbetriebnahme der Serveranwendung}
\label{ch:appendix:instructions}

\section{Einsatz während der Entwicklung}
\label{sec:appendix:instructions:development}
\begin{enumerate}
	\item \verb|./gradlew build|
	\item Das benötigte Docker Secret anlegen (siehe \autoref{sec:backend:configuration}).
	\item \verb|docker-compose up --build -d|
	\item Der Server ist nun unter \verb|http://localhost:8080| erreichbar.
\end{enumerate}
%------------------------------------------------------------------------------------------------------------------------------------------------------

\section{Einsatz in der Produktion}
\label{sec:appendix:instructions:production}
\begin{enumerate}
	\item \verb|./gradlew build|
	\item Das benötigte Docker Secret anlegen (siehe \autoref{sec:backend:configuration}).
	\item Den Wert von \verb|URL| in \verb|./environments/domain.env| auf eine valide Domäne ändern, über welche die Host-Maschine erreichbar ist.
	\item Sicherstellen, dass die Host-Maschine über die Ports 80 und 443 erreichbar ist.
	\item \verb|docker-compose -f docker-compose-prod.yml up --build -d|
	\item Der Server ist nun unter \verb|https://your.domain/koffee| erreichbar.
\end{enumerate}
%------------------------------------------------------------------------------------------------------------------------------------------------------

\chapter{Datenbankverwaltung}
\label{ch:appendix:database}

\section{Datensicherung}
\label{sec:appendix:database:backup}
\begin{lstlisting}[language=Bash, breaklines=true]
$ docker-compose exec -T mongo mongodump --archive --gzip --db koffee-database > dump.gz
\end{lstlisting}
%------------------------------------------------------------------------------------------------------------------------------------------------------

\section{Datenwiederherstellung}
\label{sec:appendix:database:restore}
\begin{lstlisting}[language=Bash, breaklines=true]
$ docker-compose exec -T mongo mongorestore --archive --gzip < dump.gz
\end{lstlisting}
%------------------------------------------------------------------------------------------------------------------------------------------------------

\chapter{Datenbankschemas}
\label{ch:appendix:schemas}

\section{User}
\label{sec:appendix:schemas:user}
\begin{table}[H]
	\begin{tabular*}{\textwidth}{l@{\extracolsep{\fill}}ll}
		Feld         & Typ                                      & Anmerkung             \\ \toprule
		id           & String                                   & Muss einzigartig sein \\ \midrule
		name         & String                                   &                       \\ \midrule
		isAdmin      & Boolean                                  & Erlaubt Authentifizierung \\ \midrule
		password     & String                                   & Optional              \\ \midrule
		transactions & List\textless{}Transaction\textgreater{} & Verwendet Polymorphie \\ \midrule
		profileImage & ProfileImage                             & Optional              \\ \bottomrule
	\end{tabular*}
	\label{tab:appendix:schemas:user}
\end{table}

\section{ProfileImage}
\label{sec:appendix:schemas:profileimage}
\begin{table}[H]
	\begin{tabular*}{\textwidth}{l@{\extracolsep{\fill}}ll}
		Feld         & Typ    & Anmerkung        \\ \toprule
		encodedImage & String & Base64-Kodierung \\ \midrule
		timestamp    & Long   &                  \\ \bottomrule
	\end{tabular*}
	\label{tab:appendix:schemas:profileimage}
\end{table}

\section{Item}
\label{sec:appendix:schemas:item}
\begin{table}[H]
	\begin{tabular*}{\textwidth}{l@{\extracolsep{\fill}}ll}
		Feld   & Typ    & Anmerkung             \\ \toprule
		id     & String & Muss einzigartig sein \\ \midrule
		name   & String &                       \\ \midrule
		amount & Int    & Optional              \\ \midrule
		price  & Double &                       \\ \bottomrule
	\end{tabular*}
	\label{tab:appendix:schemas:item}
\end{table}

\section{Transaction - Funding}
\label{sec:appendix:schemas:funding}
\begin{table}[H]
	\begin{tabular*}{\textwidth}{l@{\extracolsep{\fill}}ll}
		Feld      & Typ    & Anmerkung                      \\ \toprule
		type      & String & Muss \glqq funding\grqq{} sein \\ \midrule
		value     & Double & Wert der Guthabenaufladung     \\ \midrule
		timestamp & Long   &                                \\ \bottomrule
	\end{tabular*}
	\label{tab:appendix:schemas:funding}
\end{table}

\section{Transaction - Purchase}
\label{sec:appendix:schemas:purchase}
\begin{table}[H]
	\begin{tabular*}{\textwidth}{l@{\extracolsep{\fill}}ll}
		Feld      & Typ    & Anmerkung                       \\ \toprule
		type      & String & Muss \glqq purchase\grqq{} sein \\ \midrule
		value     & Double & Gesamtwert der Transaktion      \\ \midrule
		timestamp & Long   &                                 \\ \midrule
		itemId    & String &                                 \\ \midrule
		itemName  & String &                                 \\ \midrule
		amount    & Int    &                                 \\ \bottomrule
	\end{tabular*}
	\label{tab:appendix:schemas:purchase}
\end{table}

\section{Transaction - Refund}
\label{sec:appendix:schemas:refund}
\begin{table}[H]
	\begin{tabular*}{\textwidth}{l@{\extracolsep{\fill}}ll}
		Feld      & Typ    & Anmerkung                     \\ \toprule
		type      & String & Muss \glqq refund\grqq{} sein \\ \midrule
		value     & Double & Gesamtwert der Transaktion    \\ \midrule
		timestamp & Long   &                               \\ \midrule
		itemId    & String &                               \\ \midrule
		itemName  & String &                               \\ \midrule
		amount    & Int    &                               \\ \bottomrule
	\end{tabular*}
	\label{tab:appendix:schemas:refund}
\end{table}
%------------------------------------------------------------------------------------------------------------------------------------------------------

\chapter{Endpunkte}
\label{ch:appendix:routes}

Alle Endpunkte verwenden JSON um Daten zu senden und erhalten.
Endpunkte, welche eine zuvorige Authentifizierung voraussetzen, sind mit einem Sternchen (\verb|*|) gekennzeichnet.
Es werden keine \textit{Query}-Parameter verwendet.
Felder eventuell benötigter \textit{Body}-Parameter sind unter den jeweiligen Endpunkten aufgelistet.

\begin{description}
	\item[POST /login] Authentifiziert einen Nutzer und gibt im Erfolgsfall einen JWT zurück.
	\begin{itemize}
		\item id: String
		\item password: String
	\end{itemize}

	\item[GET /users] Gibt IDs und Namen aller Nutzer zurück.
	
	\item[*POST /users] Erstellt einen neuen Nutzer und gibt dessen ID zurück.
	\begin{itemize}
		\item id: String (Optional)
		\item name: String
		\item isAdmin: Boolean
		\item password: String (Optional)
	\end{itemize}

	\item[*PUT /users] Aktualisiert einen existierenden Nutzer.
	\begin{itemize}
		\item id: String
		\item name: String
		\item isAdmin: Boolean
		\item password: String (Optional)
	\end{itemize}

	\item[GET /users/:id] Gibt ID, Name und Guthaben des Nutzers mit der gegebenen ID zurück.
	
	\item[*DELETE /users/:id] Löscht den Nutzer mit der gegebenen ID.
	
	\item[*POST /users/:id/funding] Lädt das Guthaben des Nutzers mit der gegebenen ID um den gewünschten Betrag auf.
	\begin{itemize}
		\item amount: Double
	\end{itemize}

	\item[POST /users/:id/purchases] Kauft einen Artikel für den Nutzer mit der gegebenen ID.
	\begin{itemize}
		\item itemId: String
		\item amount: Int
	\end{itemize}

	\item[POST /users/:id/purchases/refund] Falls möglich wird der letzte Kauf des Nutzers mit der gegebenen ID storniert.
	
	\item[GET /users/:id/image] Gibt Bytes des Profilbilds eines Nutzers mit der gegebenen ID zurück.
	
	\item[GET /users/:id/image/timestamp] Antwortet mit dem Zeitstempel des Profilbilds eines Nutzers mit der gegebenen ID.
	
	\item[POST /users/:id/image] Speichert die hochgeladene Datei als Profilbild des Nutzers mit der gegebenen ID.
	\begin{itemize}
		\item Multipart Datei (kein JSON)
	\end{itemize}

	\item[DELETE /users/:id/image] Löscht das Profilbilds des Nutzers mit der gegebenen ID.
	
	
	\item[GET /items] Gibt IDs, Namen, Preise und Mengenangaben aller Artikel zurück.

	\item[*POST /items] Erstellt einen neuen Artikel und gibt dessen ID zurück.
	\begin{itemize}
		\item id: String (Optional)
		\item name: String
		\item amount: Int (Optional)
		\item price: Double
	\end{itemize}
	
	\item[*PUT /items] Aktualisiert einen existierenden Artikel.
	\begin{itemize}
		\item id: String
		\item name: String
		\item amount: Int (Optional)
		\item price: Double
	\end{itemize}
	
	\item[GET /items/:id] Gibt ID, Name, Preis und Mengenangabe des Artikels mit der gegebenen ID zurück.
	
	\item[*DELETE /users/:id] Löscht den Artikel mit der gegebenen ID.
\end{description}

\end{appendix}
